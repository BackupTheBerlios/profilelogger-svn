\documentclass[a4paper, 10pt]{article}
\usepackage{booktabs}
\usepackage[mac]{inputenc}
\usepackage{lscape}
\usepackage{geometry}

\geometry{left=1cm,right=1cm,top=1cm,bottom=1cm}

\begin{document}
\begin{landscape}
\begin{table}
\centering
\begin{tabular}{lll}
\toprule
Id & Probenbeschriftung & Beschreibung \\
\midrule
1 & jolo/HG/B22/1/1 & Top der Bank 22, ?Wurzelboden\\
\midrule
2 & jolo/HG/B26/1/1 & gesamte Bank 26, sind gewellte, dunkel laminierte Strukturen flachgedr�ckte Wurzelkan�le?\\
\midrule
3 & jolo/HG/B28/1/1a & sind dunkle Schmitze Bodenbildungsreste?\\
\midrule
4 & jolo/HG/B28/1/1b & Sind dunkle Schmitze Bodenbildungsreste?\\
\midrule
5 & jolo/HG/B32/1/1 & gewellte Laminae, ?interne Bodenoberfl�che oder flachgedr�ckter Wurzelgang\\
\midrule
6 & jolo/HG/B34/1/1 & Top der Bank. ?Wurzeln\\
\midrule
7 & jolo/HG/B34/2/1a & 2 St�cke. Top der Bank. ?Wurzeln\\
\midrule
8 & jolo/HG/B36/2/1 & 2 St�cke. ?Wurzelboden\\
\midrule
9 & jolo/HG/B39/2/1 & Top der Bank. ?Wurzelboden\\
\midrule
10 & jolo/HG/B41/2/1 & Paket mit internen Grenzfl�chen. Gips ist liegendste Schicht. Gewellte Laminae.\\
\midrule
11 & jolo/HG/B43/2/1 & dunkle basale Schicht ist Top der Bank 42. ?Wurzelboden\\
\midrule
12 & jolo/HG/B46/2/1a & Basis der Bank. Konkretionen.\\
\midrule
13 & jolo/HG/B46/2/1 & �ber jolo/HG/B46/2/1a (bei der Bergung zerbrochen). ?Konkretionen ?undeutlich gewellte Schichten\\
\midrule
14 & jolo/HG/B38+B39/2/1 & Wurzelboden. ?Durchwachsen Wurzeln die Schichtgrenze?\\
\midrule
15 & jolo/HG/B41/2/2 & Gips\\
\midrule
16 & jolo/HG/B41/2/3 & br�selige dunkle Zwischenlage. Wahrscheinlich nicht pr�parierbar.\\
\midrule
17 & jolo/HG/B41/2/4 & ?dunkle Laminae?\\
\midrule
18 & jolo/HG/B43/2/5 & Grenze zu Bank 41. ?Wurzeln\\
\midrule
19 & jolo/HG/B44+B45/2/7 & ?Konkretionen ?Intraclast Breccie?\\
\midrule
20 & jolo/HG/B45+B46/2/8 & ?Konkretionen\\
\midrule
21 & jolo/HG/B46/2/9 & Konkretionen\\
\midrule
22 & jolo/HG/B46/2/9a & Konkretionen aus Top 5 cm der Bank.\\
\midrule
23 & jolo/HG/B47/2/10 & praktisch gesamte Bank. ?Konkretionen\\
\midrule
24 & jolo/HG/B48/2/11 & praktisch gesamte Bank. ?Konkretionen\\
\midrule
25 & jolo/HG/B46/3/1 & cm gro{\ss}e Konkretionen aus der Bankmitte\\
\midrule
26 & jolo/HG/B47/3/1 & gesamte Bank. ?Konkretionen\\
\bottomrule
\end{tabular}
\caption{Samples in Profile \emph{Tesero Mb}.  Benennung der Proben: jolo/HG/Bank/Aufschluss/Laufende Nummer. jolo: Johannes Lochmann, HG: H\"ollgraben.}
\end{table}
\end{landscape}
\end{document}